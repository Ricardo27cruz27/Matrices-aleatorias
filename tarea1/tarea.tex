\documentclass[paper=letter, fontsize=14pt]{scrartcl} 

\usepackage[usenames]{color}
\usepackage[utf8]{inputenc}
%\usepackage{color}
\usepackage{graphicx}
\usepackage{epsfig}
\usepackage{multirow}
\usepackage{colortbl}
\usepackage[table]{xcolor}
\usepackage{fancyhdr}
\usepackage{graphicx}
\usepackage{graphicx}
\usepackage{verbatim}
\usepackage{pictex}  
\usepackage{multimedia}
\usepackage{listings}
\usepackage{vmargin}
\usepackage{xcolor,colortbl}
\usepackage[spanish]{babel} % language/hyphenation
\usepackage{amsmath,amsfonts,amsthm} % Math packages
\usepackage{amsbsy}
\usepackage{amssymb}
\usepackage{fancyvrb}
\usepackage{sectsty} % Allows customizing section commands
\allsectionsfont{\centering \normalfont\scshape} % Make all sections centered, the default font and small caps

\usepackage{fancyhdr} % Custom headers and footers
\pagestyle{fancyplain} % Makes all pages in the document conform to the custom headers and footers
\fancyhead{} % No page header - if you want one, create it in the same way as the footers below
\fancyfoot[L]{} % Empty left footer
\fancyfoot[C]{} % Empty center footer
\fancyfoot[R]{\thepage} % Page numbering for right footer
\renewcommand{\headrulewidth}{0pt} % Remove header underlines
\renewcommand{\footrulewidth}{0pt} % Remove footer underlines
\setlength{\headheight}{13.6pt} % Customize the height of the header

\numberwithin{equation}{section} % Number equations within sections (i.e. 1.1, 1.2, 2.1, 2.2 instead of 1, 2, 3, 4)
\numberwithin{figure}{section} % Number figures within sections (i.e. 1.1, 1.2, 2.1, 2.2 instead of 1, 2, 3, 4)
\numberwithin{table}{section} % Number tables within sections (i.e. 1.1, 1.2, 2.1, 2.2 instead of 1, 2, 3, 4)
\setpapersize{A4}
\setmargins{2.5cm}       % margen izquierdo
{2.4cm}                        % margen superior
{16.5cm}                      % anchura del texto
{23.42cm}                    % altura del texto
{10pt}                           % altura de los encabezados
{1cm}                           % espacio entre el texto y los encabezados
{0pt}                             % altura del pie de página
{2cm}                           % espacio entre el texto y el pie de página

\setlength\parindent{0pt} % Removes all indentation from paragraphs - comment this line for an assignment with lots of text

\newcommand{\horrule}[1]{\rule{\linewidth}{#1}} % Create horizontal rule command with 1 argument of height

\title{	
\normalfont \normalsize 
\textsc{Centro de Investigaci\'on en Matem\'aticas (CIMAT). Unidad Monterrey} 
\\ [25pt] 
\horrule{0.5pt} \\[0.4cm] % Thin top horizontal rule
\huge \textbf{Desigualdades.}\\ 
\horrule{2pt} \\[0.5cm] % Thick bottom horizontal rule
}

\author{Ricardo Cruz} % Your name

\date{\normalsize\today} % Today's date or a custom date


\rhead{\begin{picture}(0,0) \put(-56.7,-50){\includegraphics[width=20mm]{cimat.png}} \end{picture}}
\renewcommand{\headrulewidth}{0.5pt}

\pagestyle{fancy}

\begin{document}
\lstdefinestyle{customc}{
  belowcaptionskip=1\baselineskip,
  basicstyle=\footnotesize, 
  frame=lrtb,
  breaklines=true,
  %frame=L,
  %xleftmargin=\parindent,
  language=C,
  showstringspaces=false,
  basicstyle=\footnotesize\ttfamily,
  keywordstyle=\bfseries\color{green!40!black},
  commentstyle=\itshape\color{red!40!black},
  identifierstyle=\color{blue},
  stringstyle=\color{purple},
}

\lstset{breakatwhitespace=true,
  basicstyle=\footnotesize, 
  commentstyle=\color{green},
  keywordstyle=\color{blue},
  stringstyle=\color{purple},
  language=C++,
  columns=fullflexible,
  keepspaces=true,
  breaklines=true,
  tabsize=3, 
  showstringspaces=false,
  extendedchars=true}

\lstset{ %
  language=R,    
  basicstyle=\footnotesize, 
  numbers=left,             
  numberstyle=\tiny\color{gray}, 
  stepnumber=1,              
  numbersep=5pt,             
  backgroundcolor=\color{white},
  showspaces=false,             
  showstringspaces=false,       
  showtabs=false,               
  frame=single,                 
  rulecolor=\color{black},      
  tabsize=2,                  
  captionpos=b,               
  breaklines=true,            
  breakatwhitespace=false,    
  title=\lstname,             
  keywordstyle=\color{blue},  
  commentstyle=\color{dkgreen},
  stringstyle=\color{mauve},   
  escapeinside={\%*}{*)},      
  morekeywords={*,...}         
} 


\maketitle % Print the title


\pagebreak


Las desigualdades son una ecuación en la que tenemos una variable que no conocemos y queremos despejar su valor, usualmente se manejan las últimas letras del abecedario para denotar a dichas variables, por ejemplo, $x,y,z$, siendo la más usual $x$.\\

Para despejar el valor de la variable, debemos distinguir primero que depende de la variable y que no. Por ejemplo, si tuvieramos $-4x+5$, entonces, $-4x$ es la parte que depende de la variable, pues explicitamente se encuentra ahí. Mientras que $+5$ solo denota una $\textbf{constante}$\\

Ahora, la técnica usada para despejar la variable es a grandes rasgos, pasar todo lo que dependa de la variable de un lado de la desigualdad ($>,<,\geq,\leq$) y dejar las constantes del otro lado\\

Una estrategia que siempre funciona es ver las desigualdades como una báscula, entonces si vamos a hacer algo de un lado, tenemos que hacerlo del otro, ejemplo:

\begin{eqnarray*}
5x+3&>&3x-4\\
5x+3\textcolor{red}{-3x}&>&3x-4\textcolor{red}{-3x}\\
5x+3\textcolor{red}{-3x}&>&-4\\
\end{eqnarray*}

En esta desigualdad restamos de ambos lados $-3x$, esto para que del lado derecho se pudiera eliminar con el $3x$ que ya estaba.\\

Esto cuando ya se domina se dice que el $3x$ pasa restando al otro lado, y esto tiene sentido porque al final nos quedo restando del otro lado. Continuemos con ese ejemplo:

\begin{eqnarray*}
5x+3&>&3x-4\\
5x+3\textcolor{red}{-3x}&>&3x-4\textcolor{red}{-3x}\\
5x+3\textcolor{red}{-3x}&>&-4\\
5x+3\textcolor{red}{-3x}&>&-4\\
5x+3\textcolor{red}{-3x}\textcolor{green}{-3}&>&-4\textcolor{green}{-3}\\
5x\textcolor{red}{-3x}&>&-4\textcolor{green}{-3}\\
\end{eqnarray*}

Ahora sumamos $3$ de cada lado, para poder eliminar el $3$ que ya estaba del lado izquierdo y así quedarnos con las variables de un lado y las constantes del otro.\\

Cuando llegas a este paso es hora de sumar las constantes con constantes y hacer lo mismo con las variables:

\begin{eqnarray*}
5x+3&>&3x-4\\
5x+3\textcolor{red}{-3x}&>&3x-4\textcolor{red}{-3x}\\
5x+3\textcolor{red}{-3x}&>&-4\\
5x+3\textcolor{red}{-3x}&>&-4\\
5x+3\textcolor{red}{-3x}\textcolor{green}{-3}&>&-4\textcolor{green}{-3}\\
5x\textcolor{red}{-3x}&>&-4\textcolor{green}{-3}\\
8x&>&-7\\
\end{eqnarray*}

Y por último si la $x$ no nos queda sola, se debe quitar lo que la este multiplicando o dividiendo, haciendo algo similar a lo anterior, si esta multiplicando, pasa dividiendo:

\begin{eqnarray*}
5x+3&>&3x-4\\
5x+3\textcolor{red}{-3x}&>&3x-4\textcolor{red}{-3x}\\
5x+3\textcolor{red}{-3x}&>&-4\\
5x+3\textcolor{red}{-3x}&>&-4\\
5x+3\textcolor{red}{-3x}\textcolor{green}{-3}&>&-4\textcolor{green}{-3}\\
5x\textcolor{red}{-3x}&>&-4\textcolor{green}{-3}\\
8x&>&-7\\
\dfrac{8x}{\textcolor{blue}{8}}&>&\dfrac{-7}{\textcolor{blue}{8}}\\
x&>&\dfrac{-7}{\textcolor{blue}{8}}\\
\end{eqnarray*}


Aplicando lo que vimos arriba a un ejemplo y aplicando lo de, si tienes algo de un lado pasa con su operación inversa, entonces:


\begin{eqnarray*} 
8x-7&>&-4x+5\\
8x-7+4x&>&+5\\
8x+4x&>&+5+7\\
12x&>&12\\
x&>&\dfrac{12}{12}\\
x&>&1
\end{eqnarray*}

Analizandolo rapidamente:
\begin{itemize}
\item del primer al segundo renglon se paso el $4x$ sumando
\item del segundo al tercer renglón se paso el $7$ sumando
\item del tercer al cuarto renglón se realizaron las operaciones con solo constantes y las operaciones con solo variables
\item del cuarto al quiento renglón se paso el 12 dividiendo para dejar a la $x$ sola
\item en el último solo se realizó la división $\dfrac{12}{12}$
\end{itemize}


En general, seguir estos pasos te lleva a resolver las desigualdades, solo que existen algunas reglas más que se tienen que tener en mente, por ejemplo, si pasas dividiendo o multiplicando algo \textbf{negativo}, entonces la desigualdad se invierte
\end{document}